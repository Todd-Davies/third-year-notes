% Set the author and title of the compiled pdf
\hypersetup{
  pdftitle = {\Title},
  pdfauthor = {\Author}
}

\section{Intro}

A compiler is a program that reads another program written in one language and
translates it into an equivalent proram written in another language.

An interpreter reads the source code of a program and produces the results of
executing the source. Many issues related to compilers are also present in the
construction and execution of interpreters.

A good compiler exhibits the following qualities:

\begin{itemize}
  \item It generates correct code
  \item It generates code that runs fast
  \item It confirms to the specification of the input language
  \item It copes with any input size, number of variables, line length etc
  \item The time taken to compile source code is linear in its size
  \item It has good diagnostics
  \item It has \textit{consistent} optimisations
  \item It works will with a debugger
\end{itemize}

An example of a compiler that optimises code is if we had a for-loop such as:

\begin{verbatim}
  A = []
  for i = 0 to N:
    A[i] = i
  endfor
\end{verbatim}

If \texttt{N} was always equal to \texttt{1}, then the compiler should optimise
this to:

\begin{verbatim}
  A = [1]
\end{verbatim}

There are two things that any sensible compiler \textit{must} do:

\begin{itemize}
  \item Preserve the meaning of the program being compiled
  \item Improve the source code in some way
\end{itemize}

In addition, it could do some (it's pretty much impossible to do all) of these
things:

\begin{itemize}
  \item Make the output code run fast
  \item Make the size of the output code small
  \item Provide good feedback to the programmer; error messages, warnings etc
  \item Provide good debugging information (this is hard since transforming the
  program from one language into another often obscures the relationship between
  an instance of a program at runtime and the source code it was derived from).
  \item Compile the code quickly
\end{itemize}



