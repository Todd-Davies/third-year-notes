\documentclass{report}

\begin{document}

\begin{enumerate}
\item Didn't answer this one
\item
  \begin{enumerate}
  \item Evolutionary algorithms are driven by selection and variation.

    Selection aims to remove members of the population that have a low
    fitness, and retain individuals with a high fitness in the
    population. This can be achived through a variety of methods, but
    usually starts with scoring each individual against the value of
    an objective function, and then using a weighted random selection
    to determine which individuals to remove from the population (such
    as with a roulette wheel, stochastic ranking or a
    tournament). Essentially, selection ensures that solutions that
    are further away from minima are likely removed from the
    population, which directs the candidate solutions towards `better'
    areas of the parameter space.

    Variation tries to ensure that each member of the population is
    different. If a population is homogenous, then there is little
    point having the population in the first place, but if each member
    of the population is different (especially if there multiple
    different `species' in the population, where individuals from the
    different species score similarly highly against the objective
    function, but have different characteristics), then the solution
    space is likely better explored and the chance of finding a global
    minima is higher.
  \end{enumerate}
\end{enumerate}

\end{document}
