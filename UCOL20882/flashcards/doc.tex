\documentclass[frontgrid]{flacards}
\usepackage{color}
% For funky database symbols
\usepackage{newlfont}
\usepackage{tabularx}
\usepackage{graphicx}
\usepackage{mathtools}

\definecolor{light-gray}{gray}{0.75}

\newcommand{\frontcard}[1]{\textcolor{light-gray}{\colorbox{light-gray}{$#1$}}}
\newcommand{\backcard}[1]{#1} 

\newcommand{\flashcard}[1]{% create new command for cards with blanks
    \card{% call the original \card command with twice the same argument (#1)
        \let\blank\frontcard% but let \blank behave like \frontcard the first time
        #1
    }{%
        \let\blank\backcard% and like \backcard the second time
        #1
    }%
}

\begin{document}

\pagesetup{2}{4} 

\flashcard{Each DNA molecule is packed into a \blank{chromosome}.}

\flashcard{\blank{Genes} contain instructions for making \blank{proteins}.}

\flashcard{The two strands of DNA twist to form a \blank{double helix}.}

\flashcard{When replicating, the \blank{hydrogen bonds} between the DNA strands break, and \blank{new bases} come to bind with the exposed ones on the separated strands to form new strands.}

\flashcard{Proteins act alone or in \blank{complexes} to perform many cellular functions.}

\card{The four DNA bases are...}{Adenine, Thymine, Guanine, Cytosine}

\flashcard{A \blank{sugar-phosphate} backbone provides structure for the DNA.}

\flashcard{\blank{Hydrogen} bonds hold the two strands of DNA together.}

\flashcard{\blank{Adenine} binds to \blank{Thymine}, \blank{Cytosine} binds to \blank{Guanine.}}

\flashcard{Before a cell divides, its DNA is duplicated using \blank{semi-conservative replication}.}

\card{What is the Karyotype?}{The 23 pairs of chromosomes in the cell.}

\card{What is an autosome?}{One of the 22 pairs of normal chromosomes in humans.}

\card{In addition to the autosomes, what other chromosomes are there?}{One pair of sex chromosomes.}

\flashcard{\blank{Meiosis} is the process where a sperm producing cell or an egg producing cell makes a new cell with 23 chromosomes.}

\flashcard{\blank{Mitosis} is when an exact replica of the genome is made (46 chromosomes).}

\flashcard{\blank{Meiosis} is when only one chromosome from each pair is passed on to the new \blank{gamete} (sperm/egg).}

\flashcard{DNA $\xrightarrow[]{\blank{transcription}}$ RNA $\xrightarrow[]{\blank{translation}}$ protein}

\flashcard{When a gene is \blank{transcribed}, it forms many \blank{RNA} molecules.}

\flashcard{\blank{RNA} molecules get \blank{translated} into proteins.}

\card{Define an allele}{Any of several forms of a gene, usually arising through mutation. Alleles are responsible for hereditary variation.}

\card{Define polymorphism (in the context of DNA)}{The existence of several alleles for one gene locus. Individuals have one or two alleles per locus.}

\flashcard{\blank{Homozygous} is when a person has two copies of one allele on a gene locus.}

\flashcard{\blank{Heterozygous} is when a person has two different alleles on a gene locus.}

\flashcard{A gene is \blank{recessive} if the \blank{mutated} protein that it produces can be compensated for by the correct protein produced by \blank{an alternative allele}.}

\flashcard{If a mutated gene produces proteins that fulfil a new function, then it may be \blank{co-dominant}, since the original function will be fulfilled by \blank{the other allele}.}

\flashcard{Genes can be \blank{recessive}, \blank{dominant} or \blank{co-dominant}.}

\card{Define genotype.}{The genetic make-up of an individual, which includes the genes or alleles present in it.}

\card{Define phenotype}{The physical appearance of an individual, including its observable or measurable traits.}

\flashcard{The phenotype is controlled by \blank{proteins} derived from \blank{genes}, and the \blank{environment}.}

\card{What bloodgroup is made from two co-dominant alleles?}{AB}

\flashcard{
  Blood groups:
  \begin{tabular}{>{$}c<{$}|>{$}c<{$}>{$}c<{$}>{$}c<{$}}
        & I^A & I^B & i\\ \hline
    I^A & \blank{A}  & \blank{AB}  & \blank{A}\\
    I^B & \blank{AB} & \blank{B}   & \blank{B}\\
    i   & \blank{A}  & \blank{B}   & \blank{O}
  \end{tabular}
}

\flashcard{Allele frequency is linked to \blank{the fitness it provides} to its \blank{carriers} in a given \blank{environment}.}

\card{Define genetic fitness}{The reproductive success of a genotype, measured as the number of offspring produced by and individual that survive to a reproductive age relative to the average age for the population.}

\flashcard{If an allele provides \blank{an advantage}, it is likely to \blank{persist} and become \blank{more prominent} in a given population.}

\flashcard{Mutations have allowed us to \blank{diversify} our diet. This includes a mutation that lets us produce \blank{lactase} during adulthood (to drink milk) and another one that reduces the function of a \blank{bitter substance taste receptor} allowing us to eat broccoli and sprouts! This is an example of \blank{natural selection}.}

\flashcard{Carriers of \blank{sickle cell anaemia} alleles are \blank{asymptomatic} and get protection from malaria.}

\flashcard{Carriers of \blank{sickle cell anaemia} alleles die if they are \blank{homozygous} since their haemoglobin does not function well.}

\flashcard{People \blank{homozygous} for a mutation affecting \blank{CCR5} are asymptomatic and immune to HIV. Probably because this gave protection against \blank{the plague} and \blank{smallpox} in the past. This mutation is less effective against pathogens from \blank{developing countries}.}

\flashcard{Environment interaction can influence the genotype. \blank{Himalayan rabbits} and \blank{arctic foxes} are sensitive to temperature, and change colour at different temperatures. This is caused by temperature sensitive \blank{tyrosine}.}

\flashcard{The environment affects the phenotype; a \blank{worse diet} can make a human twin grow to be smaller, and flowers have \blank{different colours} based on the soil \blank{pH}.}

\flashcard{Most \blank{phenotypes} are due to several genes and the environment (e.g. \blank{skin colour}, \blank{height}, \blank{weight}).}

\flashcard{A greater similarity between \blank{identical twins} for a particular \blank{trait} compared to \blank{fraternal twins} provides evidence that \blank{genetic} factors play a role.}

\flashcard{\blank{Identical} twins share all their genes and their home environment. \blank{Fraternal} twins share \blank{half} their genes and a home environment.}

\card{Define a mutation}{A \textbf{permanent} alteration in the DNA sequence
passed on into daughter cells (and sometimes gametes).}

\flashcard{The size of mutations ranges from \blank{a single base pair} (\blank{single nucleotide polymorphism} - SNP) to \blank{large segments of a chromosome} (\blank{chromosome rearrangement})}

\flashcard{SNP mutations are \blank{micro-mutations}, chromosome rearrangements
are \blank{macro-mutations}}

\card{Define a hereditary mutation.}{A mutation inherited from a parent gamete and present throughout a person's life and in every cell in their body. This can be passed on to progeny through meiosis.}

\card{Define an acquired (somatic) mutation.}{When a mutation occurs at some point in a person's life, and is present only in the cell that it occurred and it's daughter cells (through mitosis).}

\card{Environmental factors that cause mutations include...}{Mutagens; chemicals, radiation etc that causes breaks between DNA bases. Biological factors such as viruses that can integrate into the genome and cause disturbances in the DNA.}

\card{Intrinsic factors causing mutations include...}{Errors during DNA replication (before mitosis) and repair. Errors during meiosis (e.g. an error in chromosome separation).}

\flashcard{Macro mutations occur during \blank{meiosis} or in \blank{late stage cancers}}

\card{Mutations during meiosis include...}{Trisomy (when a sperm has an extra chromosome) or monosomy (when a sperm has one too few chromosomes).}

% TODO: Get images for these (slide 5, lecture 2)

\card{Single chromosome macro-mutations include...}{Within one chromosome; deletion, duplication and inversion of regions of the chromosome. Within two chromosomes, part of one can go into another (insertion), parts of chromosomes can swap places (translocation).}

\flashcard{Examples of diseases caused by macro-mutations include \blank{down syndrome}, \blank{klinefelter syndrome} and \blank{Cri du chat}.}

\card{What are the three types of substitution micro-mutations and what are they caused by?}{Caused by single base substitutions (SNP), and they are silent, nonsense (STOP) and mis-sense.}

\card{How does a nonsense mutation occur?}{When a SNP (single base substitution) converts a triplet from coding a protein to coding a STOP signal.}

\card{What is a silent mutation?}{When the protein coded for by a triplet is not changed by an SNP.}

\card{What is a mis-sense mutation?}{When a SNP mutation changes the protein coded for by a triplet.}

\flashcard{\blank{Insertions and deletions} can cause great disturbances to a protein through \blank{frameshift mutations} unless the number of bases \blank{is divisible by three}, so there is no \blank{frameshift}}.

\flashcard{There are \blank{900} bad (but \blank{recessive}) alleles for cystic fibrosis. The normal gene \blank{produces enough protein to compensate}. Patient must be \blank{homozygous} for one bad allele, or \blank{heterozygous} for two.}

\flashcard{\blank{Trinucleotide repeated expansions} are when a person has many repeats of a base pair triplet. \blank{The number of repeats} dictates the likelihood of a person getting certain diseases (more is worse for the patient).}

\flashcard{Sometimes a SNP in a region far away from a gene can cause problems. In the case of lactose intolerance, a pair 13910 bases before the relevant gene is substituted (from T to C), meaning a protein cannot bind. This is recessive, since just a bit of lactase does the job.}

\flashcard{The Human Genome project took \blank{13 years} to sequence \blank{3 billion} base pairs. DNA from \blank{5 anonymous} individuals of \blank{varying ethnicity} was taken.}

\flashcard{It was discovered that humans only have \blank{20,500} genes, but it was thought that humans should have around \blank{100,000}. This was because flies have \blank{13,000} and humans are more complicated!}

\flashcard{Humans share \blank{sixty percent} of their genes with flies, and only \blank{two percent} of the human DNA codes for genes.}

\card{Why can humans get by with so few genes?}{Alternative splicing; the same gene can produce different proteins when it is shaped differently (isoforms). This means that we can make 100k proteins with 23k genes.}

\flashcard{Cells have the \blank{same genome}, but do not express the \blank{same genes and isoforms}. Where these \blank{proteins} are expressed determines the type of cell formed.}

\flashcard{Humans genomes differ by about \blank{0.01 percent}, which is about \blank{3 million} base pairs which are mostly \blank{SNP's}}

\flashcard{The frequency of SNP's is one in every \blank{300} base pairs. Most are \blank{outside genes} and have \blank{no effect on the phenotype}.}

\card{SNP's outside of genes are useful because...}{They act as landmarks for us as scientists!}

\card{GWAS stands for...}{Genome wide association studies}

\flashcard{Most diseases result from \blank{polygenic and environmental interactions}, patients with \blank{particular groups of landmark SNP's} have been found to be more at risk of developing some diseases.}

\flashcard{GWAS aim to identify the common SNP's associated with \blank{complex diseases and traits} by testing at least \blank{hundreds of thousands} of SNP's in large population samples.}

\card{Where are the samples for GWAS taken from}{Both patients who have the disease and people who do not (the control).}

\flashcard{When particular landmark SNP's are seen in greater diseased patients compared to controls, we say that the SNP's are \blank{associated} with the disease.}

\card{If a patient has SNP's associated with a disease, what does it mean?}{The patient as a higher risk of the disease (very rarely, there could be a 100 percent association).}

\flashcard{Some people will be affected more by \blank{their environment} if they have SNP's associated to a disease in their genome (e.g. are far more likely to get a disease if they smoke).}

\card{What is pharmacogenomics?}{How do patients genomes affect their response to a treatment?}

\flashcard{In 2005, \blank{less than 50} SNP's were known to be associated with diseases, in 2008, it was \blank{over 500} and now it's over \blank{14,000}.}

\card{What was the aim of the 1000 genomes project?}{To establish the most detailed catalogue of human genetic variations.}

\flashcard{On average, each person carries \blank{250-300} loss of function variants in annotated genes, and \blank{50-100} previously implicated in inherited disorders.}

\card{How many new disease causing mutations were identified in the 1000 genomes project?}{671}

\flashcard{In \blank{2014} the 100,000 genomes project was started by \blank{the NHS}. It was split between helping \blank{cancer patients} and \blank{patients with rare diseases}.}

\flashcard{The 100,000 genomes project sampled \blank{75,000} people including \blank{40,000} serious illness patients. \blank{50,000} cancer patient genomes (one cancer and one normal per patient), and \blank{50,000} rare disease genomes (three per patient; \blank{one patient genome and two blood relatives)})}

\flashcard{\blank{23andMe} and \blank{Illumina} both let you get your genome sequenced. \blank{23andMe} does not offer much advice or counselling, but \blank{illumina} does, and is therefore more expensive.}

\flashcard{Immlumina tests healthy adults interested in learning about their risk for \blank{a set of adult-onset conditions}, assessing their \blank{carrier} status and understanding their response to certain \blank{drugs}.}

\end{document} 
