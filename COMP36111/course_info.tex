\section*{Introduction}

This course unit has two objectives. The first is to introduce the student to a
range of advanced algorithms for difficult computational problems, including
matching, flow networks and linear programming. The second objective is to
outline the mathematical techniques required to analyse the complexity of
computational tasks in general. There are two pieces of assessed coursework, and
an exam at the end.

\section*{Aims}

This unit provides an advanced course in algorithms, assuming the student
already knows algorithms for common computational tasks, and can reason about
the correctness of algorithms and understand the basics of computing the
complexity of algorithms and comparing algorithmic performance.

The course focuses on the range of algorithms available for computational tasks,
considering the fundamental division of tractable tasks, with linear or
polynomial-time algorithms, and tasks that appear to be intractable, in that the
only algorithms available are exponential-time in the worst case.

To examine the range of algorithmic behaviour and this fundamental divide, two
topics are covered:

\begin{itemize}

  \item Examining a range of common computational tasks and algorithms available;
  We shall consider linear and polynomial-time algorithms for string matching
  tasks and problems that may be interpreted in terms of graphs. For the latter
  we shall consider the divide between tractable and intractable tasks, showing
  that it is difficult to determine what range of algorithms is available for
  any given task.

  \item Complexity measures and complexity classes: How to compute complexity
  measures of algorithms, and comparing tasks according to their complexity.
  Complexity classes of computational tasks, reduction techniques. Deterministic
  and non-deterministic computation. Polynomial-time classes and non-
  deterministic polynomial-time classes. Completeness and hardness. The
  fundamental classes P and NP-complete. NP-complete tasks.

\end{itemize}

\section*{Additional reading}

\begin{tabularx}{\textwidth} {|X|X|l|}
  \hline
  
  Title & Author & Year\\ \hline
    Core Algorithm design: foundations, analysis and internet examples &
    Goodrich, Michael T. and Roberto Tamassia &
    2002 \\ \hline
    Introduction to the theory of computation (3rd edition) &
    Sipser, Michael&
    2013\\ \hline
\end{tabularx}